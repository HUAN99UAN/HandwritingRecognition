%!TEX root = ../main.tex
\todo[inline]{Introduction}
\todo[inline]{Global idea, with an image}
\todo[inline]{pseudo code}

\begin{figure}
%!TEX root = ../main.tex
\MakeRobust{\Call}

\newcommand{\body}{\ensuremath{\mathit{body}}\xspace}
\newcommand{\strokewidth}{\ensuremath{\mathit{stroke\_w}}\xspace}
\newcommand{\segmentationpoints}{\ensuremath{\mathit{sps}}\xspace}
\newcommand{\segmentationpoint}{\ensuremath{\mathit{sp}}\xspace}
\newcommand{\image}{\ensuremath{\mathit{image}}\xspace}
\newcommand{\subimage}{\ensuremath{\mathit{sub\_image}}\xspace}
\newcommand{\leftsubimage}{\ensuremath{\mathit{left}}\xspace}
\newcommand{\rightsubimage}{\ensuremath{\mathit{right}}\xspace}
\newcommand{\segmentfurther}{\ensuremath{\mathit{todo}}\xspace}
\newcommand{\characters}{\ensuremath{\mathit{done}}\xspace}
\newcommand{\parameters}{\ensuremath{\mathit{parameters}}\xspace}

\begin{algorithmic}[1]
\Function{segment}{$\image,\, \parameters$}
\State \body $\gets$ \Call{body\_region}{\image}
\State \strokewidth $\gets$ \Call{stroke\_width}{\image} 
\item[]
\State \segmentationpoints $\gets$ \Call{segmentation\_points}{\body, \strokewidth} 
\State \segmentfurther \characters $\gets$ [\image], []
\item[]
\While{\Call{continue}{~}}
	\State $\subimage \gets$ \Call{select\_sub\_image}{\segmentfurther} 
	\State $\segmentationpoint \gets$ \Call{select\_sp}{\segmentationpoints} 
	\State \leftsubimage, \rightsubimage $\gets$ \Call{split}{\subimage, \segmentationpoint}
	%
   \If {\Call{is\_character}{\leftsubimage}}
      \State \Call{add}{\leftsubimage, \characters}
   \Else
		\State \Call{add}{\leftsubimage, \segmentfurther}
   \EndIf \Comment{Repeat for \rightsubimage.}
\EndWhile
\State \textbf{return} \Call{merge}{\segmentfurther, \characters}
\EndFunction
\end{algorithmic}
\caption{Binary Over Segmentation Algorithm.}\label{s:method:segmentation:algorithm}
\end{figure}

\todo[inline]{Discuss the ordering of images.}
\todo[inline]{Discuss data set parameters, outlier detection}
\todo[inline]{What are we going to do in this section}

\subsubsection{Body Region}
\todo[inline]{How do we compute the body region}
\todo[inline]{Image of wrong body region, image of correct body region.}

\subsubsection{Stroke Width}
\todo[inline]{What is the strokewidth}
\todo[inline]{How do we compute it}

\subsubsection{Segmentation Points}
\todo[inline]{Computation, based on Fig 3 in old version}
\todo[inline]{Filtering SSP, subsubsection per filter method. Show results, succesful and failure.}

\subsubsection{Termination Condition}
\todo[inline]{Wanneer stoppen we met ons lusje}
\todo[inline]{Waarom nemen we langste woord op basis van alle data, i.pv. alleen niet outliers.}

\subsubsection{Select Sub-Image}

\subsubsection{Select SSP}
\todo[inline]{How to select the next SSP }

\subsubsection{Split Image}
\todo[inline]{How to split an image, compare our approach with the splitting along a straight line. Mention A* parameters.}

\subsubsection{Segment Further}
\todo[inline]{How to determine if an image should be segmented further, or if it is an character.}