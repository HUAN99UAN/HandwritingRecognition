\documentclass[9pt,journal,compsoc]{IEEEtran/IEEEtran}
\usepackage[utf8]{inputenc}

%Language
\usepackage[english]{babel}

% Citation
% \usepackage[nocompress]{cite}
\usepackage[numbers]{natbib}

% Graphics
\usepackage[pdftex]{graphicx}
\graphicspath{{.shared/img/}{./individual/img/}}
\DeclareGraphicsExtensions{.pdf,.jpeg,.png}

% Math
\usepackage{mathtools}
\usepackage{amsmath}
\usepackage{amsfonts}
\usepackage{amssymb}
\interdisplaylinepenalty=2500

%Lay out
\usepackage{xspace}
\usepackage{booktabs}

%Pseudo Code
\usepackage{algpseudocode}

% Floats
\usepackage[caption=false,font=footnotesize,labelfont=sf,textfont=sf]{subfig}
\usepackage{stfloats}

%Hyperlinks
\usepackage{url}
\usepackage{varioref}
\usepackage{hyperref}
\usepackage[noabbrev, capitalise, nameinlink]{cleveref}
\usepackage{etoolbox}
% Patch cleverref for use with ieee, source: http://tex.stackexchange.com/a/250739
\makeatletter
\patchcmd{\@IEEEyesnumber}
  {\stepcounter}
  {\refstepcounter}
  {}{}
\patchcmd{\@@IEEEeqnarray}
  {\stepcounter}
  {\refstepcounter}
  {}{}
\patchcmd{\@@IEEEeqnarraycr}
  {\stepcounter{IEEEsubequation}}
  {\refstepcounter{IEEEsubequation}}
  {}{}
\patchcmd{\@@IEEEeqnarraycr}
  {\stepcounter{IEEEsubequation}}
  {\refstepcounter{IEEEsubequation}}
  {}{}
\patchcmd{\@@IEEEeqnarraycr}
  {\stepcounter{IEEEequation}}
  {\refstepcounter{IEEEequation}}
  {}{}
\patchcmd{\@@IEEEeqnarraycr}
  {\stepcounter{IEEEequation}}
  {\refstepcounter{IEEEequation}}
  {}{}
\makeatother

%Lists
\usepackage[inline]{enumitem}
\crefname{enumi}{stage}{stages}  
\Crefname{enumi}{Stage}{Stages}

%Math
\DeclarePairedDelimiter\abs{\lvert}{\rvert}%
\DeclarePairedDelimiter\norm{\lVert}{\rVert}
\makeatletter
\let\oldabs\abs
\def\abs{\@ifstar{\oldabs}{\oldabs*}}
\let\oldnorm\norm
\def\norm{\@ifstar{\oldnorm}{\oldnorm*}}
\makeatother

%Custom commands
\newcommand{\bigOh}[1]{\ensuremath{\mathcal{O}\left(#1\right)}}
\renewcommand{\t}[1]{\ensuremath{\mathit{#1}}}
\newcommand{\function}[2]{\textsc{#1}(#2)}

%Shortcuts
\newcommand{\astar}{A*\xspace}
\newcommand{\knnk}{\ensuremath{\mathcal{K}}\xspace}
\newcommand{\knn}{\knnk-NN\xspace}
\newcommand{\Knn}{\knnk-nearest neighbor\xspace}

%Temporary
\usepackage{layouts}
\usepackage[obeyFinal]{todonotes}
\newcommand{\angelo}[1]{\todo[color=blue!20, inline]{\textbf{Angelo: }#1}}
\newcommand{\laura}[1]{\todo[color=green!20, inline]{\textbf{Laura: }#1}}
\newcommand{\timemachine}[1]{\todo[color=red!20, inline]{\textbf{Given a time machine: }#1}}

\begin{document}

%!TEX root = /Users/laura/Repositories/HandwritingRecognition/report/main.tex
\title{Recognizing 12th Century Handwriting in the Roman Alphabet with Binary Over Segmentation}
\author{L.E.N. Baakman (S1869140)}

\markboth{Handwriting Recognition,~2015-2016}%
{Baakman: Recognizing 12th Century Handwriting in the Roman Alphabet with Binary Over Segmentation}

%!TEX root = ../main.tex
\begin{abstract}
	Off-line hand writing recognition has been an issue between researches for a long time. Various techniques have been introduce and implemented most of which are performing well for the most part. In cases however that the characters have no clear borders within the word itself, things can get more complicated. In this paper we present such a case as well as our approach to solve the problem and achieve translation of the characters provided to ASCII.
\end{abstract}

\maketitle

\IEEEdisplaynontitleabstractindextext
\IEEEpeerreviewmaketitle

\IEEEraisesectionheading{\section{Introduction}\label{s:introduction}}
%!TEX root = ../main.tex

In section \ref{ss:methods:preprocessing} we present the process of cleaning our textual data. In section \ref{ss:methods:characterSegmentation} we show how we split the words into characters whilst in section \ref{ss:methods:featureExtraction} we show the features we extracted from the characters to be used in the classification process. In section \ref{ss:methods:machineLearing} we present our machine learning approach and in section \ref{s:results} we present our results for both the characters and the words. Finally in section \ref{s:discussion} and \ref{s:conclusion} we present the discussion as well as our conclusions respectively.

\section{Methods}
\label{s:methods}
%!TEX root = ../main.tex
In this section we discuss the inner workings of our recognizer. \Cref{ss:methods:preprocessing} shortly discusses how we preprocessed our images. The next section explains how given an external segmentation we extract the character images. \Cref{ss:methods:featureExtraction} presents the method we used to represent these images as feature vectors. The classification of this representation is discussed in \cref{ss:methods:recognition}. Finally the method used for post processing is introduced in \cref{ss:methods:postprocessing}

\subsection{Preprocessing}
\label{ss:methods:preprocessing}
%!TEX root = ../../main.tex
The main goal of our preprocessing step is to remove noise, a the characters do not need to be normalized.
% Luminosity normalization
Luminosity normalization removes differences between images due to worn off ink. This step linearly scales the image to the intensity range of the $500 \times 500$ center rectangle of the image. \Cref{fig:methods:preprocessing:lumNormalization} shows the results of global thresholding with and without luminosity filter, illustrating that  without this normalization most of the text would have disappeared after the global thresholding. 
% Global thresholding
Otsu's \cite{otsu1975threshold} method is used to separate the foreground from the background. 
% Opening by Reconstruction
The last preprocessing step is the removal of small components in the image via opening with a $3 \times 3$ rectangular structuring element.



\subsection{Internal Segmentation}
\label{ss:methods:characterSegmentation}
%!TEX root = /Users/laura/Repositories/HandwritingRecognition/report/main.tex
\Cref{tab:results:loocv} shows that the validation segmentation outperforms the binary over segmentation, ideally they would have had the same performance. Below several factors that could negatively impact the performance of the binary over segmentation are discussed.

\timemachine{Plaatjes van gefaalde segmenatatie: oversegmentatie, onder segmentatie}

\subsubsection{Base Line Computation}
	As shown in \cref{fig:method:segmentation:baseline:failure} some baselines are incorrect. Consequently ligatures have more influence on the vertical pixel density histogram, this causes under segmentation of letters underneath such a ligature. A possible solution to this problem would be using a more sophisticated method to compute base lines, such as the one proposed by \citet{lee2012binary}.

\subsubsection{Segmentation Point Computation}
\timemachine{Example}
	The filtering of the segmentation lines assumes that the written alphabet contains a number of letters with holes in them and that letters are not connected by holes. This means that we can only say with certainty that the method should work for Latin alphabets. Due to noise and the preprocessing a lot of holes that should be present, are absent in the binary image. Thus the segmentation lines over these holes are not removed, which can cause over segmentation of these characters. To avoid this problem one could do hole detection on both the original preprocessed image and that image dilated with a small structuring element. The dilation should restore original holes, and also using the original image avoids missing holes that are filled by the dilation.

\subsubsection{Detecting a Character Image}
	\timemachine{Example, of the lines found on w}
	Observing the classification of sub images as either a character image or  as an image that should be segmented further, one observes that wider letters, especially the `m' and the `w' are often not recognized as letters. Consequently a word such as `woman' is often segmented as `uuonnan'. This problem could be reduced by improving the recognition of characters. One possibility would be to test how likely it is that a word is a character image or an image that should be segmented further, based on distributions of the image width, height and number of foreground pixel. If the likelihood of both classes below some threshold, the sub image, should be discarded otherwise it should be added to the most likely category. 

\subsection{Feature Extraction}
\label{ss:methods:featureExtraction}
%!TEX root = /Users/laura/Repositories/HandwritingRecognition/report/main.tex
\begin{figure}[t]
	\centering
	%!TEX root = /Users/laura/Repositories/HandwritingRecognition/report/main.tex
% \resizebox {0.6\columnwidth} {!} {
	\begin{tikzpicture}
		[
			noname/.style={%
			rectangle,
			text height=1.5ex,
			text depth=.25ex,
			text width=1em,
			text centered,
			minimum height=1em
  		}]
	    \node[anchor=south west,inner sep=0] (image) at (0,0) {\includegraphics[width=0.65\columnwidth]{individual/img/discussion/distance.png}};
	    \def\keys{{"a", "b", "c", "d", "e", "f", "g", "h", "i", "k", "l", "m", "n", "o", "p", "q", "r", "s", "t", "u", "v", "x", "y", "z"}}
	    \begin{scope}[x={(image.south east)},y={(image.north west)}]
	    % 0.04166666667 = 1/24
	        \draw[help lines,xstep=.04166666667,ystep=.04166666667] (0,0) grid (1,1);
				\foreach \x in {0,1,...,23} { 
					\node [anchor=south, noname] at (\x/24 + 1/48, 1) {\small \pgfmathparse{\keys[\x]}\pgfmathresult}; 
				}
				\foreach \y in {0,1,...,23} { 
					\node [anchor=east, noname] at (0,1 - \y/24 - 1/48) {\small \pgfmathparse{\keys[\y]}\pgfmathresult}; 
				}
	    \end{scope}
	\end{tikzpicture}
% }
	\caption{Visualization of the distance matrix of the classes with lower case letters in the train data. The distance were normalized. The white the color of a cell is the smaller the distance between the two classes is.}
	\label{fig:discussion:featureextraction:distanceMatrix}
\end{figure}
\Cref{fig:discussion:featureextraction:distanceMatrix} presents a visualization of the distance matrix of the feature vectors of the lower case letters in the train data. Each element in the matrix represent the mean normalized Euclidean distance between the feature vectors of the classes associated with that element. Ideally the diagonal would be white, and the other cells would be black, as that would indicate that there is no difference within a class and maximum difference between classes. Observing \cref{fig:discussion:featureextraction:distanceMatrix} we find that the `m' is reasonable distinguishable from the other characters. Furthermore based on the light color found in the row and column associated with the letter `k' we can also conclude that classifying this letter correctly is unlikely. The generally light color of the figure shows that the differences between classes are not good enough for classification.

The problems with the feature vectors make it hard to see discuss the recognition.

\subsection{Recognition}
\label{ss:methods:recognition}
%!TEX root = /Users/laura/Repositories/HandwritingRecognition/report/main.tex
It is hard to say something about the quality of the recognition, due the crappy
\todo[inline]{Beter k vinden}
\todo[inline]{Gezien eerdere problemen moeilijk om te bepalen dat knn het probleem is}
\todo[inline]{Andere distance measures proberen, possibly}

\subsection{Postprocessing}
\label{ss:methods:postprocessing}
%!TEX root = ../../main.tex
For post processing we build a lexicon of the words that occur in the train data, for each of these words we compute a prior probability based on their occurrence in the train data. Given some read word $w$ we compute a similarity score
\begin{equation}
	s_{\text{similarity}} = (1 - d_H(w, x)) \cdot P(x)
\end{equation}
for each entry $x$ in the lexicon. The similarity score is the normalized hamming distance between the words $w$ and $x$ multiplied with the prior probability of $x$. The lexicon entry with the highest score is assumed to be the read word.



\section{Results}
\label{s:results}
%!TEX root = ../main.tex

\begin{table*}[!t]
\caption{LOOCV results}
\label{tab:results:loocv}
\centering
	\begin{tabular}{lcccccccc}
		\toprule
					& \multicolumn{4}{c}{Validation} & \multicolumn{4}{c}{Binary}\\
					\cmidrule(r){2-5} \cmidrule(r){6-9}
		~ 			& $\mu$ 	& $\sigma$ 	& min 	& max 	& $\mu$ 	& $\sigma$ 	& min 	& max \\
		\midrule
		All 		& 0.694 	& 0.165 	& 0.3 	& 1.0 	& 0.718 	& 0.133 	& 0.25 	& 1.0 \\
		Bohemians 	& 0.731 	& 0.131 	& 0.44 	& 0.982 & 0.737 	& 0.11 		& 0.5 	& 0.982 \\
		Homilies 	& 0.647 	& 0.194 	& 0.214 & 1.0 	& 0.686 	& 0.171 	& 0.25 	& 1.0 \\
		\bottomrule
	\end{tabular}
\end{table*}

To measure how well the binary over segmentation performed we have compared this algorithm with a base line, `validation segmentation'. This baseline uses the character boundaries provided by the annotation files in the training data. When computing the error we only considered words that of which the internal segmentation was complete in the annotation file. We have defined the error of one file as the number of incorrectly classified words divided by the number of completely internally segmented words in the annotation file.  The results of this experiment are presented in \cref{tab:results:loocv}. 

We observe that the validation segmentation gets better results than the binary over segmentation. Furthermore independent of the segmentation algorithm the recognizer performs better on the pages from ``Chronicles of the Bohemians" than on pages from ``Old English Homilies".

Testing the results on previously unseen test data we find that the recognition rate for ``Chronicles of the Bohemians" is 2.28\%, for ``Old English Homilies" it is $3.88\%$ and on both books combined it is $2.56\%$. It should be noted that these results were gathered using an earlier version of the classifier that did not move segmentation points based on the width between neighboring segmentation points.

\section{Discussion}
\label{s:discussion}
%!TEX root = /Users/laura/Repositories/HandwritingRecognition/report/main.tex
The results presented in the previous section show that our recognizer did not perform very well. The sections below discuss of several of the components what might have affected their performance and how it could be improved.

\subsection{Preprocessing}
\label{ss:discussion:preprocessing}
%!TEX root = ../../main.tex
The main goal of our preprocessing step is to remove noise, a the characters do not need to be normalized.
% Luminosity normalization
Luminosity normalization removes differences between images due to worn off ink. This step linearly scales the image to the intensity range of the $500 \times 500$ center rectangle of the image. \Cref{fig:methods:preprocessing:lumNormalization} shows the results of global thresholding with and without luminosity filter, illustrating that  without this normalization most of the text would have disappeared after the global thresholding. 
% Global thresholding
Otsu's \cite{otsu1975threshold} method is used to separate the foreground from the background. 
% Opening by Reconstruction
The last preprocessing step is the removal of small components in the image via opening with a $3 \times 3$ rectangular structuring element.



\subsection{Internal Segmentation}
\label{ss:discussion:characterSegmentation}
%!TEX root = /Users/laura/Repositories/HandwritingRecognition/report/main.tex
\Cref{tab:results:loocv} shows that the validation segmentation outperforms the binary over segmentation, ideally they would have had the same performance. Below several factors that could negatively impact the performance of the binary over segmentation are discussed.

\timemachine{Plaatjes van gefaalde segmenatatie: oversegmentatie, onder segmentatie}

\subsubsection{Base Line Computation}
	As shown in \cref{fig:method:segmentation:baseline:failure} some baselines are incorrect. Consequently ligatures have more influence on the vertical pixel density histogram, this causes under segmentation of letters underneath such a ligature. A possible solution to this problem would be using a more sophisticated method to compute base lines, such as the one proposed by \citet{lee2012binary}.

\subsubsection{Segmentation Point Computation}
\timemachine{Example}
	The filtering of the segmentation lines assumes that the written alphabet contains a number of letters with holes in them and that letters are not connected by holes. This means that we can only say with certainty that the method should work for Latin alphabets. Due to noise and the preprocessing a lot of holes that should be present, are absent in the binary image. Thus the segmentation lines over these holes are not removed, which can cause over segmentation of these characters. To avoid this problem one could do hole detection on both the original preprocessed image and that image dilated with a small structuring element. The dilation should restore original holes, and also using the original image avoids missing holes that are filled by the dilation.

\subsubsection{Detecting a Character Image}
	\timemachine{Example, of the lines found on w}
	Observing the classification of sub images as either a character image or  as an image that should be segmented further, one observes that wider letters, especially the `m' and the `w' are often not recognized as letters. Consequently a word such as `woman' is often segmented as `uuonnan'. This problem could be reduced by improving the recognition of characters. One possibility would be to test how likely it is that a word is a character image or an image that should be segmented further, based on distributions of the image width, height and number of foreground pixel. If the likelihood of both classes below some threshold, the sub image, should be discarded otherwise it should be added to the most likely category. 

\subsection{Feature Extraction}
\label{ss:discussion:featureExtraction}
%!TEX root = /Users/laura/Repositories/HandwritingRecognition/report/main.tex
\begin{figure}[t]
	\centering
	%!TEX root = /Users/laura/Repositories/HandwritingRecognition/report/main.tex
% \resizebox {0.6\columnwidth} {!} {
	\begin{tikzpicture}
		[
			noname/.style={%
			rectangle,
			text height=1.5ex,
			text depth=.25ex,
			text width=1em,
			text centered,
			minimum height=1em
  		}]
	    \node[anchor=south west,inner sep=0] (image) at (0,0) {\includegraphics[width=0.65\columnwidth]{individual/img/discussion/distance.png}};
	    \def\keys{{"a", "b", "c", "d", "e", "f", "g", "h", "i", "k", "l", "m", "n", "o", "p", "q", "r", "s", "t", "u", "v", "x", "y", "z"}}
	    \begin{scope}[x={(image.south east)},y={(image.north west)}]
	    % 0.04166666667 = 1/24
	        \draw[help lines,xstep=.04166666667,ystep=.04166666667] (0,0) grid (1,1);
				\foreach \x in {0,1,...,23} { 
					\node [anchor=south, noname] at (\x/24 + 1/48, 1) {\small \pgfmathparse{\keys[\x]}\pgfmathresult}; 
				}
				\foreach \y in {0,1,...,23} { 
					\node [anchor=east, noname] at (0,1 - \y/24 - 1/48) {\small \pgfmathparse{\keys[\y]}\pgfmathresult}; 
				}
	    \end{scope}
	\end{tikzpicture}
% }
	\caption{Visualization of the distance matrix of the classes with lower case letters in the train data. The distance were normalized. The white the color of a cell is the smaller the distance between the two classes is.}
	\label{fig:discussion:featureextraction:distanceMatrix}
\end{figure}
\Cref{fig:discussion:featureextraction:distanceMatrix} presents a visualization of the distance matrix of the feature vectors of the lower case letters in the train data. Each element in the matrix represent the mean normalized Euclidean distance between the feature vectors of the classes associated with that element. Ideally the diagonal would be white, and the other cells would be black, as that would indicate that there is no difference within a class and maximum difference between classes. Observing \cref{fig:discussion:featureextraction:distanceMatrix} we find that the `m' is reasonable distinguishable from the other characters. Furthermore based on the light color found in the row and column associated with the letter `k' we can also conclude that classifying this letter correctly is unlikely. The generally light color of the figure shows that the differences between classes are not good enough for classification.

The problems with the feature vectors make it hard to see discuss the recognition.


\section{Conclusion}
\label{s:conclusion}
%!TEX root = /Users/laura/Repositories/HandwritingRecognition/report/main.tex
Based on the difference between the validation segmentation and the binary over segmentation the segmentation algorithm seems to work reasonably well. The recognizer as a whole performs badly due to non distinguishable feature vectors. It is possible that this is partly caused by insufficient preprocessing for a lest half of the data. But as the differences between the two subsets are not that big, it its quite likely that the crossings feature vectors are insufficient.

\bibliographystyle{IEEEtranN}
\bibliography{biblio}

%!TEX root = /Users/laura/Repositories/HandwritingRecognition/report/main.tex
\begin{IEEEbiographynophoto}{L.E.N. Baakman}
implemented the used framework and the recognizer. She also designed and implemented the segmentation algorithm. Furthermore she optimized the code written for the pre-processing, feature extraction and post-processing. This paper was completely authored by her, it should be noted that \cref{sss:introduction:offline:preprocessing}, \ref{sss:introduction:offline:representation}, \ref{ss:methods:preprocessing} and \ref{ss:methods:featureExtraction} are (partly) based on text provided by E. Karountzos.
\end{IEEEbiographynophoto}

\begin{IEEEbiographynophoto}{E. Karountzos}
set up the initial versions of the pre-processing, feature extraction and post-processing. He also wrote the text that \cref{sss:introduction:offline:preprocessing}, \ref{sss:introduction:offline:representation}, \ref{ss:methods:preprocessing} and \ref{ss:methods:featureExtraction} are (partly) based on.
\end{IEEEbiographynophoto}

\end{document}


